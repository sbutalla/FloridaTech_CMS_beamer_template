\documentclass[10pt]{beamer}
\usetheme{metropolis}
\usepackage{xcolor}
\usepackage{booktabs}
\usepackage[scale=2]{ccicons}
\usepackage{graphicx}
\usepackage{adjustbox}
\usepackage{transparent}
\usepackage{eso-pic}
\graphicspath{	{images/}	}
\usepackage{amsmath}
\usepackage{url}
\usepackage{hyperref}
\usepackage{listings}
\lstset{language=Python,
    basicstyle=\ttfamily\bfseries,
    commentstyle=\color{red}\itshape,
  showstringspaces=false,
  keywordstyle=\color{blue}\bfseries}
\setbeamertemplate{bibliography item}[text]

\DeclareMathOperator\erf{erf}

%Defines Florida Tech red color
\definecolor{FTRed}{RGB}{120,0,0}
\setbeamercolor{progress bar}{fg=FTRed}
\setbeamercolor{frametitle}{fg=white,bg=FTRed}

%Hides section slides
\metroset{sectionpage=none}

\usepackage{pgfplots}
\usepgfplotslibrary{dateplot}

\usepackage{xspace}
\newcommand{\themename}{\textbf{\textsc{metropolis}}\xspace}

 \hypersetup{
     colorlinks=true,
     linkcolor=blue,
     filecolor=blue,
     citecolor = black,      
     urlcolor=FTRed,
     }


\title{Your Title}
\subtitle{Your Subtitle}
\date{Date}
\author{Author Name}
\institute{}
\titlegraphic{
	\begin{picture}(0,0)
    \put(70,-175){\makebox(0,0)[rt]{\includegraphics[height=3cm]{FloridaTechNewLogo.png}}}
  \end{picture}
\begin{picture}(0,0)
    \put(320,-160){\makebox(0,0)[rt]{\includegraphics[height=3cm]{CMSlines.png}}}
  \end{picture}
  \begin{picture}(0,0)
    \put(320,10){\makebox(0,0)[rt]{\includegraphics[height=2.2cm]{cern_logo.png}}}
  \end{picture}
  
 
 }
%\hspace{9.5cm}\vspace{-0.5cm} 
\setbeamertemplate{footline}[text line]{%
  \parbox{\linewidth}{\vspace{-0.15cm}\hspace{4cm} Author Names  -- ``Footer Title'' -- Date \hfill\insertpagenumber}}


\begin{document}


\maketitle

%Adds Florida Tech logo to each page after this command
\addtobeamertemplate{frametitle}{}{%
\begin{tikzpicture}[remember picture,overlay]
\node[anchor=south west, yshift=-4pt] at (current page.south west) {\includegraphics[height=1.2cm]{FloridaTechNewLogo}};
\end{tikzpicture}}
\addtobeamertemplate{frametitle}{}{%
\begin{tikzpicture}[remember picture,overlay]
\node[anchor=north east,yshift=2pt,xshift=1.5pt] at (current page.north east) {\includegraphics[height=0.8cm]{CMS-Color-Var1.pdf}};
\end{tikzpicture}}
\addtobeamertemplate{frametitle}{}{%
\begin{picture}(0,0)
    \put(305,29.2){\makebox(0,0)[rt]{\includegraphics[height=0.8cm]{cern_logo_1}}}
  \end{picture}}

%\begin{frame}{Table of contents}
%        \begin{columns}
%            \begin{column}{.5\textwidth}
%             \setbeamertemplate{section in toc}[sections numbered]
  %              \tableofcontents[sections=1- 6]
  %              
     %       \end{column}
        %    \begin{column}{.5\textwidth}
           %  \setbeamertemplate{section in toc}[sections numbered]
           %     \tableofcontents[sections=7-11]
           % \end{column}
       % \end{columns}
   % \end{frame}


%Default ToC
%\begin{frame}{Table of contents}
 %\setbeamertemplate{section in toc}[sections numbered]
  %\tableofcontents[hideallsubsections]
%\end{frame}

%\begin{frame}[fragile]{ME0 Update}

%\vspace{-0.4cm}
%\begin{columns}
%\begin{column}{0.4\textwidth}
%\begin{itemize}\footnotesize{
%\item Teed a line to the main N2 line in the cleanroom and routed to the storage rack
%\item Removed the ME0 from the X-ray box and placed in storage}
%\end{itemize}
%\end{column}
%\begin{column}{0.4\textwidth}
%\center
%\vspace{-0.4cm}
%\includegraphics[width=5cm]{M5moduleXrayBox.jpg}\\
%\end{column}
%\end{columns}
%\end{frame}


\begin{frame}[fragile]{Frame with columns}
\vspace{-0.5cm}
\begin{columns}
\begin{column}{0.5\textwidth}
\begin{itemize}\scriptsize{
\itemsep0em
\item Column 0}
\end{itemize}
\end{column}
\begin{column}{0.5\textwidth}
\begin{itemize}\scriptsize{
\itemsep0em
\item Column 1}
\end{itemize}
\end{column}
\end{columns}
\end{frame}

\begin{frame}[fragile]{Frame with figure}

\center
\includegraphics[width=3cm]{CMS-Color-Var1.pdf}\\
\scriptsize{The CMS logo.}\\


\end{frame}

\begin{frame}[fragile]{Frame with equation}

$$i\gamma^{\mu}\partial_{\mu}\psi-\psi=0$$
\end{frame}

\end{document}



